\documentclass[xcolor=dvipsnames]{beamer}
\usetheme{Dresden}
\usepackage{beamerthemeshadow}
\usepackage[utf8]{inputenc}
\usepackage{listings}
\usepackage{tabularx}
\usepackage{graphicx}

%%%
%%% Frame Settings
%%%

\setbeamercolor{normal text}{fg=white,bg=black!90}
\setbeamercolor{structure}{fg=white,bg=black}
\setbeamercolor{alerted text}{fg=red!85!black}
\setbeamercolor{item projected}{use=item,fg=black,bg=item.fg!35}
\setbeamercolor*{palette primary}{use=structure,fg=structure.fg}
\setbeamercolor*{palette secondary}{use=structure,fg=structure.fg!95!black}
\setbeamercolor*{palette tertiary}{use=structure,fg=structure.fg!90!black}
\setbeamercolor*{palette quaternary}{use=structure,fg=structure.fg!95!black,bg=black!80}
\setbeamercolor*{title}{fg=white,bg=black}
\setbeamercolor*{framesubtitle}{fg=white,bg=black}
\setbeamercolor*{block title}{parent=structure,bg=black!60}
\setbeamercolor*{block body}{fg=black,bg=black!10}
\setbeamercolor*{block title alerted}{parent=alerted text,bg=black!15}
\setbeamercolor*{block title example}{parent=example text,bg=black!15}

%%%
%%% lstlistin Settings
%%%

\makeatletter
\patchcmd{\lsthk@SelectCharTable}{`)}{``}{}{} % patch listings
%\patchcmd{\lsthk@SelectCharTable}{``}{`)}{}{} % undo patch if needed
\makeatother

\definecolor{codeGreen}{rgb}{0.05,0.7,0.05}
\definecolor{codeOrange}{rgb}{0.8,0.5,0}
\definecolor{codeBlue}{rgb}{0,0.2,1.0}
\definecolor{codePurple}{rgb}{0.5,0,0.8}

\lstdefinestyle{C++}{
	backgroundcolor=\color{black},
	basicstyle=\footnotesize,
	breakatwhitespace=false,
	breaklines=true,
	captionpos=b,
	commentstyle=\color{codeOrange},
	deletekeywords={...},
	escapeinside={\%*}{*)},
	extendedchars=true,
	frame=single,
	gobble=4,
	keepspaces=true,
	keywordstyle=\color{codeBlue},
	language=C++,
	morekeywords={NULL, nullptr,...},
	numbers=left,
	numbersep=6pt,
	numberstyle=\tiny\color{white},
	rulecolor=\color{gray},
	showspaces=false,
	showstringspaces=false,
	showtabs=false,
	stepnumber=1,
	stringstyle=\color{yellow},
	tabsize=4,
	emph={proc,retp,endp,local}, emphstyle={\color{blue}\textbf},
	literate=	{->}{{{\color{codeGreen}-$>$}}}2
				{=}{{{\color{codeGreen}=}}}1
				{!}{{{\color{codeGreen}!}}}1
				{;}{{{\color{codeGreen};}}}1
				{(}{{{\color{codeGreen}(}}}1
				{)}{{{\color{codeGreen})}}}1
				{[}{{{\color{codeGreen}[}}}1
				{]}{{{\color{codeGreen}]}}}1
				{\{}{{{\color{codeGreen}\{}}}1
				{\}}{{{\color{codeGreen}\}}}}1
}

\lstdefinestyle{Bash}{
	backgroundcolor=\color{black},
	basicstyle=\footnotesize,
	breakatwhitespace=false,
	breaklines=true,
	breakindent=0pt,
	captionpos=b,
	commentstyle=\color{codeOrange},
	deletekeywords={...},
	escapeinside={\%*}{*)},
	extendedchars=true,
	frame=single,
	keepspaces=true,
	keywordstyle=\color{codeBlue},
	language=Bash,
	morekeywords={NULL, nullptr,...},
	rulecolor=\color{gray},
	showspaces=false,
	showstringspaces=false,
	showtabs=false,
	stepnumber=1,
	stringstyle=\color{yellow},
	tabsize=4,
	literate=	{\$}{{{\color{codePurple}\$}}}1
}

%%%
%%% Übersicht
%%%
\title{Simulation of Microservices to improve Benchmarking}
\subtitle{Fachstudie}
\author{Samuel Beck, Johannes Günthör, Christoph Zorn}
\institute{RSS}
\date{18.05.2017}

%%%
%%% Inhalt
%%%
\begin{document}
	
%%%
%%%	Folie 1
%%%
\begin{frame}
	
	\titlepage 
	
\end{frame}

%%%
%%%	Folie 2
%%%
\section{Overview}
\subsection{Overview}
\begin{frame}
	\frametitle{Overview}
	
	\begin{itemize}
		\item Who are we?
		\item Where do we stand?
		\begin{itemize}
			\item Our preexisting know how
			\item Paper research
			\item Tool research
		\end{itemize}
		\item Schedule
		\item Questions?
	\end{itemize}
	
\end{frame}

%%%
%%%	Folie 3
%%%
\section{Who are we?}
\subsection{Who are we?}
\begin{frame}
	\frametitle{Who are we?}
	
	\begin{itemize}
		\item Samuel Beck (Bsc. Softwaretechnik)\newline\newline
		\item Johannes Günthör (Bsc. Softwaretechnik)\newline\newline
		\item Christoph Zorn (Bsc. Softwaretechnik)\newline\newline
	\end{itemize}	
	
\end{frame}


%%%
%%%	Folie 4
%%%
\section{Where do we stand?}
\begin{frame}
	\frametitle{Where do we stand?}
	
\end{frame}


%%%
%%%	Folie 5
%%%
\subsection{Our preexisting know how}
\begin{frame}
	\frametitle{Our preexisiting know how}
	
	Coding knowledge:
	\begin{itemize}
		\item Java
		\item C/C++
		\item HTML5/JS/PHP/MySQL
		\item Python
	\end{itemize}
	
	Working experience:
	\begin{itemize}
		\item Moog since 04/16
		\item Daimler since 05/16
	\end{itemize}
	
	(completed projects: ie: Sopra/Stupro)
	
\end{frame}


%%%
%%%	Folie 6
%%%
\subsection{Paper research}
\begin{frame}
	\frametitle{Paper research}
	
	\begin{itemize}
		\item CASPA
		\item Model driven generation of...
		\item performance simultion of runtime...
		\item performance engineering for microservices ...\newline
		\item Seminar: Resilience Assessment of Software ...
	\end{itemize}
	
\end{frame}


%%%
%%%	Folie 6
%%%
\subsection{Tool research}
\begin{frame}
	\frametitle{Tool research}
	
	\begin{itemize}
		\item spigo\newline
	\end{itemize}
	
	Achievments:
	\begin{itemize}
		\item create a microservice architecture
		\item create visualization of an architecture
		\item inject chaosmonkey into architecture
		\item generate 'Workflow' of the architecture
	\end{itemize}
	
\end{frame}


%%%
%%%	Folie 7
%%%
\section{Schedule}
\subsection{Schedule}
\begin{frame}
	\frametitle{Schedule}
	
	\begin{tabular}{l|l} 
		May & Presentation, spigo, metric research \\
		& \\
		June & spigo enhancement/implementation \\
		& \\
		July & implementation \\
		& \\
		August & exam phase \\
		& \\
		September & exam phase, implementation \\
		& \\
		October & implementation \\
		& \\
		November & polishing, implementation \\
	\end{tabular}
\end{frame}

%%%
%%%	Folie 8
%%%
\section{Questions?}
\subsection{Questions?}
\begin{frame}
	\frametitle{Questions?}
	
	\begin{LARGE}
		\begin{center}
			Any questions left?
		
			-
		
			Thanks for your patience!
		\end{center}
	\end{LARGE}
	
\end{frame}

\end{document}
