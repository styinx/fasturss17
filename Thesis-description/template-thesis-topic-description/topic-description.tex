\documentclass[a4paper,12pt]{article}

\usepackage{titlesec}
\titleformat*{\section}{\large\bfseries}
\titleformat*{\subsection}{\normalsize\bfseries}

\pagestyle{empty}

%\usepackage[ngerman]{babel}

\usepackage[utf8]{inputenc}
\usepackage[top=0.8in, bottom=0.8in, left=0.8in, right=0.8in]{geometry}

\usepackage{graphicx}
\usepackage{url}
\usepackage{paralist}
\usepackage{lmodern}
\usepackage[authoryear]{natbib}
\usepackage{blindtext}
\renewcommand\textbullet{\ensuremath{\bullet}}

\renewcommand{\familydefault}{\sfdefault}

\date{}

\title{
%\vspace*{-2cm} \includegraphics[width=16cm]{header.png}
\includegraphics[width=6cm]{figures/stuttgart-vector.pdf}\hfill{\includegraphics[width=3cm]{figures/rss_logo.pdf}}
\quad \\ [0.5cm]
{\large \textit{Fachstudie (Bachelor Softwaretechnik):}} \\ [1mm]
{\Large Simulation-based resilience prediction of microservice architectures}
}

\begin{document}
	

\maketitle

\thispagestyle{empty}

\vspace{-2.5cm}


\subsection*{Background and Motivation}
	Testing was, is and will be an important part of software development. Current state of the art testing in large microservice based architectures is done in “live” environments. Front runner for these testing methods is Netflix using their self build tool-set Simian Army \cite{NSA}. The downside of testing during the runtime of a system is that potential failures can cost a lot of money by affecting users of the system. The lack of non-live testing tools for microservices has led to a demand for a prediction simulator \cite{CLOUD}.


\subsection*{Goals}
	The goal of this work is developing a software tool that is able to simulate microservice architectures, resilience anti-patterns and failures of one or more instances of microservices. The first step is to look at multiple existing software simulators and analyze them on their ability to be modified into a microservice simulator. Based on this approach we will decide to extend an existing tool or build our own from scratch. The resulting outcome will lead to the creation of a simulator. The resulting software solution will be able to test microservice architectures before release and deployment.

\begin{scriptsize}
\bibliographystyle{abbrv}
\bibliography{bibliography.bib}
\end{scriptsize}

\subsection*{Contact}
Examiner and Supervisor: Dr.-Ing. André van Hoorn, van.hoorn@informatik.uni-stuttgart.de \\
University of Stuttgart, Inst.\ for Software Technology, Reliable Software Systems Group \\ \newline
Supervisor: M.Sc. Thomas Düllmann, duellmann@informatik.uni-stuttgart.de \\
University of Stuttgart, Inst.\ for Software Technology, Reliable Software Systems Group \\
\end{document}

